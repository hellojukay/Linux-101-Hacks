%-*- coding UTF8 -*-
% Chapter1.tex
% author: hellojukay

\documentclass[UTF8]{ctexart}

\title{第一章:强大的 cd 命令}
\author{hellojukay}

\usepackage{listings}
\usepackage{color}
\usepackage{xcolor}

\definecolor{dkgreen}{rgb}{0,0.6,0}
\definecolor{gray}{rgb}{0.5,0.5,0.5}
\definecolor{mauve}{rgb}{0.58,0,0.82}

\lstset{frame=tb,
  language=bash,
  aboveskip=3mm,
  belowskip=3mm,
  showstringspaces=false,
  columns=flexible,
  basicstyle={\small\ttfamily},
  numbers=none,
  numberstyle=\tiny\color{gray},
  keywordstyle=\color{blue},
  commentstyle=\color{dkgreen},
  stringstyle=\color{mauve},
  breaklines=true,
  breakatwhitespace=true,
  tabsize=4
}


\begin{document}
\maketitle
cd 命令是 unix 里面使用频率最高的命令之一,本章讲介绍 6 个非常有用
的技巧来提高我们切换目录的效率。

\section*{技巧1:定义 CDPATH 环境变量}
cd 命令最常用的就是进入某个文件夹,然后进入他的子目录,定义 CDPATH 环
境变量,能让我们不指定完整目录的情况下直接进入某个目录,见如下:
\begin{lstlisting}
    hellojukay@local linux101hack $ cd apt
    bash: cd: apt: 没有那个文件或目录
    hellojukay@local linux101hack $ export CDPATH=/etc
    hellojukay@local linux101hack $ cd apt
    /etc/apt
    hellojukay@local apt $ pwd
    /etc/apt
    hellojukay@local apt $ 
\end{lstlisting}
看起来很有效果,将他加入到 \$HOME/.bashrc 中去,和 PATH 环境变量的配置很
相似,你也可以添加到个目录到 CDPATH 中,就像这样:
\begin{lstlisting}
    export CDPATH=$CDPATH:/var:/usr:~
\end{lstlisting}
\end{document}