%-*- coding UTF8 -*-
% Chapter2.tex
% author: hellojukay

\documentclass[UTF8]{ctexart}

\title{第一章:时间日期相关操作}
\author{hellojukay}

\usepackage{listings}
\usepackage{color}
\usepackage{xcolor}

\definecolor{dkgreen}{rgb}{0,0.6,0}
\definecolor{gray}{rgb}{0.5,0.5,0.5}
\definecolor{mauve}{rgb}{0.58,0,0.82}

\lstset{frame=tb,
  language=bash,
  aboveskip=3mm,
  belowskip=3mm,
  showstringspaces=false,
  columns=flexible,
  basicstyle={\small\ttfamily},
  numbers=none,
  numberstyle=\tiny\color{gray},
  keywordstyle=\color{blue},
  commentstyle=\color{dkgreen},
  stringstyle=\color{mauve},
  breaklines=true,
  breakatwhitespace=true,
  tabsize=4
}
\begin{document}

\section*{技巧7:设置系统日期和时间}

改变系统的日期和时间,使用如下命令(修改之前最好手动关闭 NTP 自动同步:sudo timedatectl set-ntp false):
\begin{lstlisting}
# 修改系统日期,时间需要使用 root 权限
date {mmddhhmiyyyy.ss}
\end{lstlisting}
\begin{itemize}
    \item mm 月份
    \item dd 日期
    \item hh 小时
    \item mi 分钟
    \item yyyy 年份
    \item ss 秒钟
\end{itemize}
例如:要将日期设置为 2008 年 1 月 31 日 10 点
19 分 53 秒:
\begin{lstlisting}
± |master U:1 ?:2 ✗| → sudo date 0131192008.53                                                                                                                               
2008年 01月 31日 星期四 19:20:53 CST

 2008-01-31 19:20:56 ⌚  local in ~/github/Linux-101-Hacks
± |master U:1 ?:2 ✗| → date
2008年 01月 31日 星期四 19:20:57 CST
    
\end{lstlisting}
你也可以向下面一样,通过给定参数的方式来设置日期:
\begin{lstlisting}
date 013122192009.53
date +%Y%m%d -s "20090131"
date -s "01/31/2009 22:19:53"
date -s "31 JAN 2009 22:19:53"
\end{lstlisting}
只修改时间,不修改日期:
\begin{lstlisting}
date +%T -s "22:19:53"
date +%T%p -s "10:19:53PM"
\end{lstlisting}

\begin{quote}
    \small{自动同步日期和时间,可以使用 NTP 服务:sudo timedatectl set-ntp true}
\end{quote}


\section*{技巧8: 修改硬件日期和时间}

设置系统硬件时间之前,请确保当前系统时间已经设置正确,设置硬件的日期和
时间使用如下命令:
\begin{lstlisting}
hwclock –systohc
hwclock --systohc –utc
\end{lstlisting}
不加任何参数调用 hwclock 命令表示查看当前硬件的日期和时间
\begin{lstlisting}
# 如果你不是 root 用户, 请加上 sudo 
hwclock
\end{lstlisting}
检查当前硬件时间是否设置为 UTC 时间使用如下命令:
\begin{lstlisting}
# 如果文件不存在,则可以手动创建之
cat /etc/sysconfig/clock
ZONE="America/Los_Angeles"
UTC=false
ARC=false
\end{lstlisting}


\section*{技巧9: 格式化显示日期和时间}
这里有好几种方法格式化显示日期和时间:
\begin{lstlisting}
2019-07-29 09:56:58 ⌚  local in ~/github/Linux-101-Hacks                                                                                                                    
± |master U:1 ?:2 ✗| → date                                                                                                                                                   
2019年 07月 29日 星期一 10:00:09 CST                                                                                                                                          
                                                                                                                                                                              
 2019-07-29 10:00:09 ⌚  local in ~/github/Linux-101-Hacks                                                                                                                    
± |master U:1 ?:2 ✗| → date --date="now"                                                                                                                                      
2019年 07月 29日 星期一 10:00:20 CST                                                                                                                                          
                                                                                                                                                                              
 2019-07-29 10:00:20 ⌚  local in ~/github/Linux-101-Hacks
2019-07-29 09:56:58 ⌚  local in ~/github/Linux-101-Hacks                                                                                                                    
± |master U:1 ?:2 ✗| → date                                                                                                                                                   
2019年 07月 29日 星期一 10:00:09 CST                                                                                                                                          
                                                                                                                                                                              
 2019-07-29 10:00:09 ⌚  local in ~/github/Linux-101-Hacks                                                                                                                    
± |master U:1 ?:2 ✗| → date --date="now"                                                                                                                                      
2019年 07月 29日 星期一 10:00:20 CST                                                                                                                                          
                                                                                                                                                                              
 2019-07-29 10:00:20 ⌚  local in ~/github/Linux-101-Hacks                                                                                                                    
± |master U:1 ?:2 ✗| → date --date="today"
2019年 07月 29日 星期一 10:00:33 CST

 2019-07-29 10:00:33 ⌚  local in ~/github/Linux-101-Hacks
± |master U:1 ?:2 ✗| → date --date='1970-01-01 00:00:01 UTC +5 hours' +%s
18001

± |master U:1 ?:2 ✗| → date '+Current Date: %m/%d/%y%nCurrent Time:%H:%M:%S'
Current Date: 07/29/19
Current Time:10:00:57
\end{lstlisting}
不同的命令行选择含义如下:
\begin{itemize}
    \item \%D : 日期(月/日/年)
    \item \%d : 月份中的天数(01..31)
    \item \%m : 一年中的月份(01..12)
    \item \%y : 两位数的年份(00..99)
    \item \%a : 本地化的星期简写(一..日)
    \item \%A : 本地化的星期全称(星期一...星期日)
    \item \%b : 本地化的月份简写(1月..12月)
    \item \%B : 本地化的月份全称(一月..十二月)
    \item \%H : 24小时制小时(01..23)
    \item \%I : 12小时制度小时(01..11)
    \item \%Y : 四位数年份(2019)
\end{itemize}


\section*{技巧10: 展示过去的日期和时间}
一下方式能够展示过去的日期和时间:
\begin{lstlisting}
date --date='3 seconds ago'
date --date="1 day ago"
date --date="2 days ago"
date --date="1 month ago"
date --date="2 months ago"
date --date="1 year ago"
date --date="2 years ago"
date --date="yesterday"
date --date="10 months 2 day ago"
\end{lstlisting}


\section*{技巧11: 展示将来的日期和时间}
一下方式能够展示将来的日期和时间:
\begin{lstlisting}
date --date='3 seconds'
date --date='4 hours'
date --date='tomorrow'
date --date="1 day"
date --date="2 days"
date --date='1 month'
date --date='2 months'
date --date='1 week'
date --date='2 weeks'
date --date="2 years"
date --date="next day"
date --date="-1 days ago"
date --date="this Wednesday"
\end{lstlisting}
\end{document}