%-*- coding UTF8 -*-
% Chapter4.tex
% author: hellojukay

\documentclass[UTF8]{ctexart}

\title{第三章: 常用命令}
\author{hellojukay}

\usepackage{listings}
\usepackage{color}
\usepackage{xcolor}

\definecolor{dkgreen}{rgb}{0,0.6,0}
\definecolor{gray}{rgb}{0.5,0.5,0.5}
\definecolor{mauve}{rgb}{0.58,0,0.82}

\lstset{frame=tb,
  language=bash,
  aboveskip=3mm,
  belowskip=3mm,
  showstringspaces=false,
  columns=flexible,
  basicstyle={\small\ttfamily},
  numbers=none,
  numberstyle=\tiny\color{gray},
  keywordstyle=\color{blue},
  commentstyle=\color{dkgreen},
  stringstyle=\color{mauve},
  breaklines=true,
  breakatwhitespace=true,
  tabsize=4
}

\begin{document}
\section*{技巧17: grep}
grep 命令常用来做文本搜索,他支持需要强大的命令行选项.语法格式如下:
\begin{lstlisting}
grep [options] pattern [files]  
\end{lstlisting}
\textbf{如何在文件中找到包含指定关键字的所有行?}\newline
这个例子我们将在 /etc/passwd 文件中搜索关键字 hellojukay
\begin{lstlisting}
2019-08-05 09:17:32 ⌚  local in ~/github/Linux-101-Hacks
± |master ✓| → grep hellojukay /etc/passwd
hellojukay:x:1000:1000::/home/hellojukay:/bin/bash       
\end{lstlisting}
使用 -v 选项则匹配搜索不包含关键字的行,这个例子我们在 /et/passwd 文件
中搜索不包含 hellojukay 的所有行:
\begin{lstlisting}
2019-08-05 09:27:45 ⌚  local in ~/github/Linux-101-Hacks                                                                                                                    
± |master ?:1 ✗| → grep hellojukay /etc/passwd -v                                                                                                                             
root:x:0:0:root:/root:/bin/bash                                                                                                                                               
daemon:x:1:1:daemon:/usr/sbin:/usr/sbin/nologin                                                                                                                               
bin:x:2:2:bin:/bin:/usr/sbin/nologin                                                                                                                                          
sys:x:3:3:sys:/dev:/usr/sbin/nologin                                                                                                                                          
sync:x:4:65534:sync:/bin:/bin/sync                                                                                                                                            
games:x:5:60:games:/usr/games:/usr/sbin/nologin                                                                                                                               
man:x:6:12:man:/var/cache/man:/usr/sbin/nologin                                                                                                                               
lp:x:7:7:lp:/var/spool/lpd:/usr/sbin/nologin                                                                                                                                  
mail:x:8:8:mail:/var/mail:/usr/sbin/nologin                                                                                                                                   
news:x:9:9:news:/var/spool/news:/usr/sbin/nologin                                                                                                                             
uucp:x:10:10:uucp:/var/spool/uucp:/usr/sbin/nologin                                                                                                                           
proxy:x:13:13:proxy:/bin:/usr/sbin/nologin                                                                                                                                    
www-data:x:33:33:www-data:/var/www:/usr/sbin/nologin                                                                                                                          
backup:x:34:34:backup:/var/backups:/usr/sbin/nologin                                                                                                                          
list:x:38:38:Mailing List Manager:/var/list:/usr/sbin/nologin                                                                                                                 
irc:x:39:39:ircd:/var/run/ircd:/usr/sbin/nologin                                                                                                                              
gnats:x:41:41:Gnats Bug-Reporting System (admin):/var/lib/gnats:/usr/sbin/nologin                                                                                             
nobody:x:65534:65534:nobody:/nonexistent:/usr/sbin/nologin                                                                                                                    
systemd-network:x:100:102:systemd Network Management,,,:/run/systemd/netif:/usr/sbin/nologin  
\end{lstlisting}
\textbf{统计文件中有多少行符合匹配规则的文本}\newline
这里例子统计在 /etc/passwd 文件中有多少文件包含 hellojukay 关键字:
\begin{lstlisting}
2019-08-05 09:29:17 ⌚  local in ~/github/Linux-101-Hacks
± |master ?:1 ✗| → grep hellojukay /etc/passwd -c
1    
\end{lstlisting}
当然,可以使用 -vc 命令来统计不包含指定关键字的文本的行数总和。\newline
\textbf{忽略大小写检索搜索文本}\newline
使用 -i 选项能够无视大小写搜索字符串。
\begin{lstlisting}
    # grep -i john /etc/passwd
    jsmith:x:1082:1082:John Smith:/home/jsmith:/bin/bash
    jdoe:x:1083:1083:John Doe:/home/jdoe:/bin/bash
\end{lstlisting}
\textbf{如何递归搜索子文件夹中的文件?}\newline
使用 -r 选了能够递归搜索子文件夹中文件,当然, -r 选项也能和以上所述选项一起使用,
-ri , -rv 等等


\section*{技巧18: find}
find 命令也是经常会用在文件检索上的命令,看下他的使用语法:
\begin{lstlisting}
  find [pathnames] [conditions]
\end{lstlisting}
\textbf{如何找到文件名中包含指定关键字的文件?}\newline
这个例子展示如何在 /etc 目录中查找文件中包含 ssh 的文件:
\begin{lstlisting}
  2019-08-05 09:54:05 ⌚  local in ~/github/Linux-101-Hacks
  ± |master ?:2 ✗| → sudo find /etc -name "ssh"
  /etc/init.d/ssh
  /etc/default/ssh
  /etc/ssh  
\end{lstlisting}
\textbf{如何查找文件体积大于某个数值的文件?}\newline
这个例子展示查找 \$HOME 目录下体积大于100M的文件:
\begin{lstlisting}
   2019-08-05 09:57:13 ⌚  local in ~/github/Linux-101-Hacks
± |master ?:2 ✗| → sudo find $HOME -size +100M
gradle/caches/5.4.1/generated-gradle-jars/gradle-api-5.4.1.jar
\end{lstlisting}
\textbf{如何根据文件修改时间来查找文件?}\newline
这个例子展示查找 60 天前下载的文件:
\begin{lstlisting}
  2019-08-08 12:59:02 ⌚  local in ~/github/Linux-101-Hacks
  ± |master U:1 ✗| → find /home/hellojukay/Downloads/ -mtime +60
  /home/hellojukay/Downloads/
  /home/hellojukay/Downloads/SwitchyOmega_Chromium (1).crx
  /home/hellojukay/Downloads/SwitchyOmega_Chromium.crx
  /home/hellojukay/Downloads/Lanxin_Setup_6.5.15.600668.exe
  /home/hellojukay/Downloads/code_1.30.2-1546901646_amd64.deb
  /home/hellojukay/Downloads/goland-2018.3.2.tar.gz
  /home/hellojukay/Downloads/FireShot
  /home/hellojukay/Downloads/sonar-scanner-cli-3.0.3.778.zip
  /home/hellojukay/Downloads/shadowsocks-deepin_1.2.2_amd64.deb
  /home/hellojukay/Downloads/CentOS-7-x86_64-DVD-1810.iso  
\end{lstlisting}
这个例子展示查找 3 天内下载的文件:
\begin{lstlisting}
  2019-08-08 13:00:36 ⌚  local in ~/github/Linux-101-Hacks
  ± |master U:1 ✗| → find /home/hellojukay/Downloads/ -mtime -3
  /home/hellojukay/Downloads/
  /home/hellojukay/Downloads/thinkocaml.pdf  
\end{lstlisting}
\textbf{如何找 tar.gz 结尾的文件,并且删除他们?}
\begin{lstlisting}
  find / -type f -name *.tar.gz -exec ls -l {} \;
\end{lstlisting}
\textbf{如何根据日期找到文件,并且按照日期压缩归档}
\begin{lstlisting}
find /home/jsmith -type f -mtime +60 | xargs tar -cvf
/tmp/`date '+%d%m%Y'_archive.tar`
\end{lstlisting}


\section*{技巧19:标准输出与标准错误输出}
\textbf{将标准输出重定向} \newline
一些时候,我们执行脚本的时候,可能不想看到脚本的标准输出或者错误输出,
我们可以将输出重定向到 /dev/null
\begin{lstlisting}
  Suppress standard output using > /dev/null
\end{lstlisting}
这个技巧在你 debugging 的时候非常有用:
\begin{lstlisting}
  ./start.sh > /dev/null
\end{lstlisting}
\textbf{将标准错误输出重定向} \newline
当你不想看到标准错误输出的时候,这个功能也同样非常实用
\begin{lstlisting}
  ./shell-script.sh 2> /dev/null
\end{lstlisting}


\section*{技巧20:join命令}
join 命令能合并两个文件中的字段,比如:我们将员工列表文件和工资列表文件合并:
\begin{lstlisting}
  $ cat employee.txt
100
200
300
400
Jason Smith
John Doe
Sanjay Gupta
Ashok Sharma
$ cat bonus.txt
100
200
300
400
$5,000
$500
$3,000
$1,250
$ join employee.txt bonus.txt
100
200
300
400
Jason Smith $5,000
John Doe $500
Sanjay Gupta $3,000
Ashok Sharma $1,250
\end{lstlisting}


\section*{技巧21:拼写大小写转换}
将文件内容转换称大写
\begin{lstlisting}
hellojukay@local Linux-101-Hacks (master) $ cat hello.txt 
hello world
hellojukay@local Linux-101-Hacks (master) $ tr a-z A-Z < hello.txt 
HELLO WORLD
hellojukay@local Linux-101-Hacks (master) $ 
\end{lstlisting}
将文件内容转称小写
\begin{lstlisting}
  hellojukay@local Linux-101-Hacks (master) $ cat hello.txt 
  HELLO WORD
  hellojukay@local Linux-101-Hacks (master) $ tr A-Z a-z < hello.txt 
  hello word
  hellojukay@local Linux-101-Hacks (master) $ 
\end{lstlisting}


\section*{技巧22: xargs 命令}
xargs 是一个非常强大的命令行工具,它能将数据传给命令行参数。

当你使用 rm 删除许多文件的时候,可能会出现参数太长。你可以使用 xargs 命令来避免这个问题。
\begin{lstlisting}
  find ~ -name '.log -print0 | xargs -I 0 rm -f 
\end{lstlisting}
你可以或者 /etc/ 目录下的所有的 *.conf 文件
\begin{lstlisting}
  find /etc/ -name *.conf | xargs ls -l
\end{lstlisting}
列出所有的文件地址,并且下载他们
\begin{lstlisting}
  cat url.txt | xargs wget -c
\end{lstlisting}
找到 jpg 图片,并且归档他们
\begin{lstlisting}
  find / -name *.jpg -type f -print | xargs tar -cvzf image.tar.gz 
\end{lstlisting}
复制所有的图片到外置存储\footnote{可以直接 cp *.jpg /external-hard/dir}
\begin{lstlisting}
  ls *jpg | xargs -n1 -i cp {} /external-hard/dir
\end{lstlisting}


\section*{技巧23: sort 命令}
sort 命令排序文件中每一行,这里有一些例子向你展示如何使用 sort 命令进行排序。

按照字符串升序排列
\begin{lstlisting}
  sort name.txt
\end{lstlisting}
按照字符串降序排列
\begin{lstlisting}
  sort -r name.txt
\end{lstlisting}
\end{document}