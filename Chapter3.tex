%-*- coding UTF8 -*-
% Chapter3.tex
% author: hellojukay

\documentclass[UTF8]{ctexart}

\title{第三章: ssh client 使用技巧}
\author{hellojukay}

\usepackage{listings}
\usepackage{color}
\usepackage{xcolor}

\definecolor{dkgreen}{rgb}{0,0.6,0}
\definecolor{gray}{rgb}{0.5,0.5,0.5}
\definecolor{mauve}{rgb}{0.58,0,0.82}

\lstset{frame=tb,
  language=bash,
  aboveskip=3mm,
  belowskip=3mm,
  showstringspaces=false,
  columns=flexible,
  basicstyle={\small\ttfamily},
  numbers=none,
  numberstyle=\tiny\color{gray},
  keywordstyle=\color{blue},
  commentstyle=\color{dkgreen},
  stringstyle=\color{mauve},
  breaklines=true,
  breakatwhitespace=true,
  tabsize=4
}

\begin{document}


\section*{技巧12: 查看 ssh client 版本}
有些时候可能需要查看当前系统运行的 ssh client 的版本,可以使用 ssh -V (大写V)来查看版本
情况,需要说明的是一般系统自带的 ssh client 都是 openssh 实现方式,例如我的 deepin 系统
就是使用的 openssh:
\begin{lstlisting}
± |master U:1 ?:1 ✗| → ssh -V
OpenSSH_7.7p1 Debian-2, OpenSSL 1.0.2o  27 Mar 2018
\end{lstlisting}
也有一些系统会使用 ssh2 的实现:
\begin{lstlisting}
$ ssh -V
ssh: SSH Secure Shell 3.2.9.1 (non-commercial version)
on i686-pc-linux-gnu
\end{lstlisting}


\section*{技巧13: 使用 ssh 登录远程服务器}
当你第一次从本地登录远程服务器的时候,命令行展示 host key 没有找到,会让你输入 yes 才能继续,
输入以后 host key 会被保存在本地的 \$HOME/.ssh/known\_host 文件中,当你第二次登录服务器的时候
便不会在让你输入 yes 了,而是直接输入密码
\begin{lstlisting}
2019-07-30 13:26:24 ⌚  local in ~/github/Linux-101-Hacks
± |master U:1 ?:1 ✗| → ssh licong@10.249.175.54
Password:
\end{lstlisting}
如果当服务器的 host key 发生了改变,你在登录服务器的时候
可能会收到警告提示信息,可能触发提示信息的原因有两个:
\begin{itemize}
    \item 有人升级了服务器的操作系统,或者重新安装了 openssh
    \item 有人在服务器上执行了恶意操作等
\end{itemize}
这时候时候在输入 yes 之前,最好是先联系系统管理员,查明原因。


\section*{技巧14:调试 ssh 链接}
有些时候我们在诊断问题的时候希望能够看到服务器登录的调试系统,使用 ssh -v (小写v)选项
来查看登录的 debug 日志信息
\begin{lstlisting}
ssh licong@10.95.58.31 -v 
\end{lstlisting}


\section*{技巧15: 本地与远程服务器 ssh 终端来回切换}
有时候我们通过 ssh 登录要远程服务器,这个时候我们要在本地机器上执行一条命令,我们不得不
退出登录,执行完毕以后,再次登录远程服务器。事实上,我们不必这样做,我们不需要断开 ssh 链接
的会话,只需要秩序一下步骤即可:
\begin{enumerate}
  \item ssh 登录远程服务器 ssh licong@10.95.55.207
  \item 输入 \char`\~
  \item 按下 ctrl z
\end{enumerate}
这个时候你的终端应该已经自动切换到本地机器了
\begin{lstlisting}
  [licong@analysis01 ~]$ ~^Z [suspend ssh]

  [1]+  已停止               ssh licong@10.95.55.207  
\end{lstlisting}
查看本地任务,有一个暂停的 ssh 任务
\begin{lstlisting}
  ± |master U:1 ✗| → jobs 
  [1]+  已停止               ssh licong@10.95.55.207  
\end{lstlisting}
如果想要再次回到远程服务器的终端,只需要激活该任务即可
\begin{lstlisting}
fg %1
\end{lstlisting}


\section*{技巧16:统计 ssh 链接流量}
当我们在使用 SSH2 的时候,我们可以通过特殊的的指令查看链接的流量信息,
在终端输入  \char`\~s 即可:
\begin{lstlisting}
remotehost$ [Note: The ~s is not visible on the
command line when you type.]
remote host: remotehost
local host: localhost
remote version: SSH-1.99-OpenSSH_3.9p1
local version: SSH-2.0-3.2.9.1 SSH Secure
Shell (non-commercial)
compressed bytes in: 1506
uncompressed bytes in: 1622
compressed bytes out: 4997
uncompressed bytes out: 5118
packets in: 15
packets out: 24
rekeys: 0
Algorithms:
Chosen key exchange algorithm: diffie-hellman-
group1-sha1
Chosen host key algorithm: ssh-dss
Common host key algorithms: ssh-dss,ssh-rsa
Algorithms client to server:
Cipher: aes128-cbc
MAC: hmac-sha1
Compression: zlib
\end{lstlisting}
\end{document}